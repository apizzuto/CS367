\documentclass[11pt,letterpaper]{article}
\usepackage[utf8]{inputenc}
\usepackage[english]{babel}
\usepackage{titlesec}
%%%%%%%%%%%%%%%%%%%%%%%%%%%%%%%%%%%%%%%%%%%%%%%%%%%%%
\usepackage{amsmath}
\usepackage{amsfonts}
\usepackage{amssymb}
\usepackage{mathtools}
\usepackage[margin=1in]{geometry}
%%%%%%%%%%%%%%%%%%%%%%%%%%%%%%%%%%%%%%%%%%%%%%%%%%%%%
\usepackage{graphicx}
\usepackage{tikz}
\usepackage{tikz-3dplot}
%%%%%%%%%%%%%%%%%%%%%%%%%%%%%%%%%%%%%%%%%%%%%%%%%%%%%
\usepackage{varioref}
\usepackage{fancyref}
\usepackage{float}
\floatstyle{boxed}
\restylefloat{figure}
\usepackage{framed}
%%%%%%%%%%%%%%%%%%%%%%%%%%%%%%%%%%%%%%%%%%%%%%%%%%%%%
\usepackage{listings}
\usepackage{color}

\titleformat{\subsection}[runin]
  {\normalfont\large\bfseries}{\thesubsection}{1em}{}
\titleformat{\subsubsection}[runin]
  {\normalfont\normalsize\bfseries}{\thesubsubsection}{1em}{}

\definecolor{dkgreen}{rgb}{0,0.6,0}
\definecolor{gray}{rgb}{0.5,0.5,0.5}
\definecolor{mauve}{rgb}{0.58,0,0.82}

\lstset{language=Java,
  aboveskip=3mm,
  belowskip=3mm,
  showstringspaces=false,
  columns=flexible,
  basicstyle={\small\ttfamily},
  numbers=none,
  numberstyle=\tiny\color{gray},
  keywordstyle=\color{blue},
  commentstyle=\color{dkgreen},
  stringstyle=\color{mauve},
  breaklines=true,
  breakatwhitespace=true,
  tabsize=3
}
%%%%%%%%%%%%%%%%%%%%%%%%%%%%%%%%%%%%%%%%%%%%%%%%%%%%%
%Script R%
\usepackage{calligra}
\DeclareMathAlphabet{\mathcalligra}{T1}{calligra}{m}{n}
\DeclareFontShape{T1}{calligra}{m}{n}{<->s*[2.2]callig15}{}
\newcommand{\scripty}[1]{\ensuremath{\mathcalligra{#1}}}
\newcommand{\sr}{\scripty{r}}
\newcommand{\vsr}{\vec{\sr\,}}
%%%%%%%%%%%%%%%%%%%%%%%%%%%%%%%%%%%%%%%%%%%%%%%%%%%%%
%Macros%
\newcommand{\dint}[2]{\int\limits_{#1}^{#2}}

%%%%%%%%%%%%%%%%%%%%%%%%%%%%%%%%%%%%%%%%%%%%%%%%%%%%%
\author{Alex Pizzuto}
\title{CS 367 Homework 2}
\begin{document}
\date{}
\maketitle
\hrule

\section{Question One}
We are starting with a \texttt{LinkedList} class implemented as a doubly-linked chain of nodes with a dummy header node. The class uses the class \texttt{DblListnode} and has the following fields:
\begin{lstlisting}
	private DblListnode<E> items;
	private int numItems;
\end{lstlisting}
We have been given the following code to reverse the order of the objects in a specified sublist:
\begin{lstlisting}
	public void reverse(int pos1, int pos2){
	  // If pos1 == pos2, reversing a single list element does nothing
	  // If pos1 > pos2, reversing an expty sublist does nothing
	  if (pos1 >= pos2)
	     return;
									
	  // We swap the 1st and last items in the sublist,
	  //  then recursively reverse the remaining sublist
	  // We stop when the remaining sublist has size 0 or 1

	  // Swap list items at pos1 and pos2
	  E temp = remove(pos2);
	  add(pos2, get(pos1));
	  remove(pos1);
	  add(pos1, temp);

	  // Now recursively reverse remainer of sublist (if any)
	  // The remaining sublist is from pos1+1 to pos2-1
	  reverse(pos1+1, pos2-1);
	}
\end{lstlisting}
And we will now complete a second version of this method that directly changes the chain of nodes by unlinking the nodes to be swapped and re-links them into the chain in the appropriate way using only DblListnode methods.
\begin{lstlisting}
	/**
	     * Reverses the order of the items from pos1 to pos2 
	     * NOTE: The Double Linked List will have a header node that is null.
	     * 
	     * @param pos1 The start position of the reversal group
	     * @param pos2 The end position of the reversal group
	     */
	    public void reverse(int pos1, int pos2){
	  	  	if(pos1 < 0 || pos2 < 0 || pos1 > numItems - 1 || pos2> numItems - 1) {
	  	  		throw new IndexOutOfBoundsException();
	  	  	}
	    		if (pos1 >= pos2 ) {
	    			return; //return if input doesn't make sense
	    		}
    		
	    		//Access the node at pos1
	    		DblListnode<E> pos1node = items.getNext(); 
	    		// items is the header node so this is the first node in the list
	    	    for (int k = 0; k < pos1; k++) {
	    	    		pos1node = pos1node.getNext();
	    	    }
    	    
	    	    //Access the node at pos2
	    	    DblListnode<E> pos2node = pos1node.getNext();
	    	    //To get the second node, start from the first
	    	    for (int k = pos1 + 1; k < pos2; k++ ) {
	    	    		pos2node = pos2node.getNext();
	    	    }
    	    
	    	    //If pos1 is the beginning of the list, it still has a next and prev
	    	    //If pos2 is at the end of the list, it only has a prev
	    	    DblListnode<E> node1Prev = pos1node.getPrev();
	    	    DblListnode<E> node1Next = pos1node.getNext();
	    	    DblListnode<E> node2Prev = pos2node.getPrev();
    	    
	    	    //treat pos2 at the end of the list as a special case
	    	    if (pos2 == numItems - 1) {	
	    	    		if(pos2-pos1 == 1) {
	    	    			//deal with nodes next to each other
	    	    			pos1node.setNext(null);
	    	    			pos1node.setPrev(pos2node);
	    	    			node1Prev.setNext(pos2node);
	    	    			pos2node.setNext(pos1node);
	    	    			pos2node.setPrev(node1Prev);
	    	    		} else {
	    	    			pos1node.setNext(null);
	    	    			pos1node.setPrev(node2Prev);
	    	    			node1Prev.setNext(pos2node);
	    	    			node1Next.setPrev(pos2node);
	    	    			pos2node.setNext(node1Next);
	    	    			pos2node.setPrev(node1Prev);
	    	    			node2Prev.setNext(pos1node);
	    	    		}
	    	    }
    	    
	    	    //Deal with all other cases
	    	    else {
	    	    		DblListnode<E> node2Next = pos2node.getNext();
	    	    		if(pos2-pos1 == 1) {
	    	    			//deal with nodes next to each other
	    	    			pos1node.setNext(node2Next);
	    	    			pos1node.setPrev(pos2node);
	    	    			node1Prev.setNext(pos2node);
	    	    			pos2node.setNext(pos1node);
	    	    			pos2node.setPrev(node1Prev);
	    	    			node2Next.setPrev(pos1node);
	    	    		} else {
	    	    			pos1node.setNext(node2Next);
	    	    			pos1node.setPrev(node2Prev);
	    	    			pos2node.setNext(node1Next);
	    	    			pos2node.setPrev(node1Prev);
	    	    			node1Next.setPrev(pos2node);
	    	    			node1Prev.setNext(pos2node);
	    	    			node2Next.setPrev(pos1node);
	    	    			node2Prev.setNext(pos1node);
	    	    		}
	    	    }
	    	    reverse(pos1+1, pos2-1);

	  	}
\end{lstlisting}
\section{Question Two} 
We can find the worst-case time complexity for the method from the previous question in terms of \texttt{N}, the list size. 

All of the test statements for positions in the beginning are clearly \texttt{O(1)}. When we obtain pos1node and pos2node, we could potentially have to traverse the entire list, so this step is \texttt{O(N)}. From there, all we are doing is constant time operations. However, that is only one of the recurisve iterations. Worst case, we could have \texttt{pos1} reference the first node in the list and \texttt{pos2} reference the last node in the list. In this case, we must perform \texttt{N/2} iterations. Although each successive iteration requires 2 less operations to obtain the original nodes, we can still bound the time complexity by \texttt{N} iterations of \texttt{O(N)} operations, thus it is quadratic in time, or \texttt{O(N$^2$)}.

\end{document}
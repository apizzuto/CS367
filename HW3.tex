\documentclass[11pt,letterpaper]{article}
\usepackage[utf8]{inputenc}
\usepackage[english]{babel}
\usepackage{titlesec}
%%%%%%%%%%%%%%%%%%%%%%%%%%%%%%%%%%%%%%%%%%%%%%%%%%%%%
\usepackage{amsmath}
\usepackage{amsfonts}
\usepackage{amssymb}
\usepackage{mathtools}
\usepackage[margin=1in]{geometry}
%%%%%%%%%%%%%%%%%%%%%%%%%%%%%%%%%%%%%%%%%%%%%%%%%%%%%
\usepackage{graphicx}
\usepackage{tikz}
\usepackage{tikz-3dplot}
%%%%%%%%%%%%%%%%%%%%%%%%%%%%%%%%%%%%%%%%%%%%%%%%%%%%%
\usepackage{varioref}
\usepackage{fancyref}
\usepackage{float}
\floatstyle{boxed}
\restylefloat{figure}
\usepackage{framed}
%%%%%%%%%%%%%%%%%%%%%%%%%%%%%%%%%%%%%%%%%%%%%%%%%%%%%
\usepackage{listings}
\usepackage{color}

\titleformat{\subsection}[runin]
  {\normalfont\large\bfseries}{\thesubsection}{1em}{}
\titleformat{\subsubsection}[runin]
  {\normalfont\normalsize\bfseries}{\thesubsubsection}{1em}{}

\definecolor{dkgreen}{rgb}{0,0.6,0}
\definecolor{gray}{rgb}{0.5,0.5,0.5}
\definecolor{mauve}{rgb}{0.58,0,0.82}

\lstset{language=Java,
  aboveskip=3mm,
  belowskip=3mm,
  showstringspaces=false,
  columns=flexible,
  basicstyle={\small\ttfamily},
  numbers=none,
  numberstyle=\tiny\color{gray},
  keywordstyle=\color{blue},
  commentstyle=\color{dkgreen},
  stringstyle=\color{mauve},
  breaklines=true,
  breakatwhitespace=true,
  tabsize=3
}
%%%%%%%%%%%%%%%%%%%%%%%%%%%%%%%%%%%%%%%%%%%%%%%%%%%%%
%Script R%
\usepackage{calligra}
\usepackage{qtree}
\DeclareMathAlphabet{\mathcalligra}{T1}{calligra}{m}{n}
\DeclareFontShape{T1}{calligra}{m}{n}{<->s*[2.2]callig15}{}
\newcommand{\scripty}[1]{\ensuremath{\mathcalligra{#1}}}
\newcommand{\sr}{\scripty{r}}
\newcommand{\vsr}{\vec{\sr\,}}
%%%%%%%%%%%%%%%%%%%%%%%%%%%%%%%%%%%%%%%%%%%%%%%%%%%%%
%Macros%
\newcommand{\dint}[2]{\int\limits_{#1}^{#2}}

%%%%%%%%%%%%%%%%%%%%%%%%%%%%%%%%%%%%%%%%%%%%%%%%%%%%%
\author{Alex Pizzuto}
\title{CS 367 Homework 3}
\begin{document}
\date{}
\maketitle
\hrule

\section{Question One}
We assume that general trees are implemented using a \texttt{Treenode} class that includes the following fields and methods:
\begin{lstlisting}
	// fields
	private T data;
	private List<Treenode<T>> children;
 
	// methods
	public T getData() { return data; }
	public List<Treenode<T>> getChildren() { return children; }
\end{lstlisting}
With this, we will write a method, \texttt{isBinary}, with the following header:
\begin{lstlisting}
	public boolean isBinary( Treenode<T> n )
\end{lstlisting}
that will determine if the general tree rooted by \texttt{n} is also a binary tree. Our method is as follows:
\begin{lstlisting}
	public boolean isBinary( Treenode<T> n ) {
		//First, we define a base case
		boolean value = true;
		if (n.getChildren.size() == 0 ) {
			return value;
		}
		if (n.getChildren.size() > 2 ) {
			return false;
		}
		Iterator<T> itrt = n.getChildren.iterator();
		while (itrt.hasNext() ) {
			value = value && isBinary(itrt.next());
		}
		return value;
	}	
\end{lstlisting}


\section{Question Two}
Next, we write the \texttt{findNegatives} method for a binary tree implemented using a \texttt{BinaryTreenode} class with the following header:
\begin{lstlisting}
	public static List<Integer> findNegatives( BinaryTreenode<Integer> n)
\end{lstlisting}
Our method will return a list containing all the negative values in a binary tree containing \texttt{Integer} data. First, we establish the rules for recursion of this function:
\begin{itemize}
	\item The list of negative values in an empty tree is the empty list.
	\item The list of negative values in a tree with one node is a list containing the node's data if the node's data is negative
	\item The list of negative values in a tree with more than one node is the concatenation of lists of the negative values in each subtree rooted by the node's children.
\end{itemize}

With this, we can write the method (assuming an \texttt{ArrayList} implementation of a \texttt{List}).
\begin{lstlisting}
	public static List<Integer> findNegatives( BinaryTreenode<Integer> n) { 
		List<Integer> negs = new ArrayList<Integer>();
		if (n.getData() instanceof Integer && n.getData() < 0 ) {
			negs.add(n.getData());
		}
		if(n.getLeft() != null) {
			negs.addAll(findNegatives(n.getLeft()));
		}
		if(n.getRight() != null ) {
			negs.addAll(findNegatives(n.getRight()));
		}
		return negs;	
	}
\end{lstlisting}


\section{Question Three}
We assume that binary search trees are implemented using a BSTnode class that includes the following fields and methods:
\begin{lstlisting}
// fields
private K key;
private BSTnode<K> left, right;
 
// methods
public K getKey() { return key; }
public BSTnode<K> getLeft()  { return left; }
public BSTnode<K> getRight() { return right; }
public void setLeft(BSTnode<K> newL)  { left  = newL; }
public void setRight(BSTnode<K> newR) { right = newR; }
\end{lstlisting}
where K is a class that implements the \texttt{Comparable} interface. With this, we write the \texttt{secondSmallest} method:
\begin{lstlisting}
	public K secondSmallest(BSTnode<K> n) {
		BSTnode<K> tmp;
		while(n.getLeft() != null ) {
			if(n.getLeft().getLeft != null ){
				n = n.getLeft(); //If there are more grandchildren, move down the tree
			}else if(n.getLeft().getRight() == null ) {
				return n; 
				//If both the left and right child of the left child are null,
				//then return the parent
			} else if (n.getLeft().getRight() != null) {
				n = n.getLeft().getRight();
				while(n.getLeft() != null ) {
					n = n.getLeft();
				}
				return n;
				//If the left child of the left child is null, but the left child
				//has a right child, return the smallest value of the tree
				//rooted by the right child of the left child
			}
		}
	}
\end{lstlisting}










\end{document}
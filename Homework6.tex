\documentclass[11pt,letterpaper]{article}
\usepackage[utf8]{inputenc}
\usepackage[english]{babel}
\usepackage{titlesec}
%%%%%%%%%%%%%%%%%%%%%%%%%%%%%%%%%%%%%%%%%%%%%%%%%%%%%
\usepackage{amsmath}
\usepackage{amsfonts}
\usepackage{amssymb}
\usepackage{mathtools}
\usepackage[margin=1in]{geometry}
%%%%%%%%%%%%%%%%%%%%%%%%%%%%%%%%%%%%%%%%%%%%%%%%%%%%%
\usepackage{graphicx}
\usepackage{tikz}
\usetikzlibrary{calc}
\usepackage{tikz-3dplot}
%%%%%%%%%%%%%%%%%%%%%%%%%%%%%%%%%%%%%%%%%%%%%%%%%%%%%
\usepackage{varioref}
\usepackage{fancyref}
\usepackage{float}
\floatstyle{boxed}
\restylefloat{figure}
\usepackage{framed}
%%%%%%%%%%%%%%%%%%%%%%%%%%%%%%%%%%%%%%%%%%%%%%%%%%%%%
\usepackage{listings}
\usepackage{color}

\titleformat{\subsection}[runin]
  {\normalfont\large\bfseries}{\thesubsection}{1em}{}
\titleformat{\subsubsection}[runin]
  {\normalfont\normalsize\bfseries}{\thesubsubsection}{1em}{}

\definecolor{dkgreen}{rgb}{0,0.6,0}
\definecolor{gray}{rgb}{0.5,0.5,0.5}
\definecolor{mauve}{rgb}{0.58,0,0.82}

\lstset{language=Java,
  aboveskip=3mm,
  belowskip=3mm,
  showstringspaces=false,
  columns=flexible,
  basicstyle={\small\ttfamily},
  numbers=none,
  numberstyle=\tiny\color{gray},
  keywordstyle=\color{blue},
  commentstyle=\color{dkgreen},
  stringstyle=\color{mauve},
  breaklines=true,
  breakatwhitespace=true,
  tabsize=3
}
%%%%%%%%%%%%%%%%%%%%%%%%%%%%%%%%%%%%%%%%%%%%%%%%%%%%%
%Script R%
\usepackage{calligra}
\usepackage{qtree}
\DeclareMathAlphabet{\mathcalligra}{T1}{calligra}{m}{n}
\DeclareFontShape{T1}{calligra}{m}{n}{<->s*[2.2]callig15}{}
\newcommand{\scripty}[1]{\ensuremath{\mathcalligra{#1}}}
\newcommand{\sr}{\scripty{r}}
\newcommand{\vsr}{\vec{\sr\,}}
%%%%%%%%%%%%%%%%%%%%%%%%%%%%%%%%%%%%%%%%%%%%%%%%%%%%%
%Macros%
\newcommand{\dint}[2]{\int\limits_{#1}^{#2}}

%%%%%%%%%%%%%%%%%%%%%%%%%%%%%%%%%%%%%%%%%%%%%%%%%%%%%
\author{Alex Pizzuto}
\title{CS 367 Homework 6}
\begin{document}
\date{}
\maketitle
\hrule

\section{Question One}
We are given the following incomplete definition of the \texttt{Graphnode} class:
\begin{lstlisting}
class Graphnode<T> {
 
    private boolean visitMark;
    private List<Graphnode<T>> successors;
 
    public boolean getVisitMark() {
        return visitMark;
    }
 
    public void setVisitMark(boolean mark) {
        visitMark = mark;
    }
 
    public List<Graphnode<T>> getSuccessors() {
        return successors;
    }
}
\end{lstlisting}
And we write the \texttt{hasSelfCycle} method:
\begin{lstlisting}
public boolean hasSelfCycle( Graphnode<T> node ) {
	return hasSelfCycle(node, node);
}

public boolean hasSelfCycle(Graphnode<T> node, Graphnode<T> dest) {
	node.setVisitMark(true);
	
	for(Graphnode<t> i : node.getSuccessors()) {
		if (i == dest) {
			return true;
		} else if (!i.getVisitMark()) {
			if(hasSelfCycle(i, dest) {
				return true; //Check to see if successor leads back to dest
			}
		}
	}
	return false; //If we make it here, return false
}
\end{lstlisting}

\section{Question Two}

We trace Dijkstra's algorithm on the provided graph. 



\begin{table}[h]
	\begin{center}
		\begin{tabular}{|c|c|} \hline
	\shortstack{visited nodes and their \\ shortest distances from start} & \shortstack{dist values for nodes in U \\(only finite values, listed in increasing order)} \\ \hline \hline
	- & (0,S) \\ \hline
	S: 0 & (4,G), (11,H), (33,P) \\ \hline
	S: 0, G: 4 & (10,R), (11,H), (11,P)  \\ \hline
	S: 0, G: 4, R: 10 & (11,H), (11,P), (30,A) \\ \hline
	S: 0, G: 4, R: 10, H: 11 & (11,P), (30,A) \\ \hline
	S: 0, G: 4, R: 10, H: 11, P: 11 & (13,A) \\ \hline
	S: 0, G: 4, R: 10, H: 11, P: 11, A: 13 & empty \\ \hline
		\end{tabular}
	\end{center}
\end{table}


\section{Question Three}
Our three topological orderings are:
\begin{itemize}
	\item A E B F C D
	\item C B D F E A
	\item C A D B E F
\end{itemize}




\end{document}